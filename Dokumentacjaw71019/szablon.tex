% **** Szablon pracy magisterskiej, licencjackiej lub inżynierskiej ****

\documentclass[polish,12pt,twoside,a4paper]{report}



\input{style.tex}

%definicja przydatnych poleceń
\newcommand{\wydzial}{KOLEGIUM INFORMATYKI STOSOWANEJ}
\newcommand{\kierunek}{Kierunek: INFORMATYKA}
\newcommand{\specjalnosc}{Specjalność: {Inżynieria Danych (ID)}}
\newcommand{\autor}{Casper Czech}
\newcommand{\album}{Nr albumu studenta w71019}
\newcommand{\temat}{System obsługi parkingu}
\newcommand{\promotor}{mgr inż. Ewa Żesławska}
\newcommand{\typpracy}{Praca projektowa programowanie obiekotwe C\#}
\newcommand{\miasto}{Rzeszów}
\newcommand{\rok}{2025}

\begin{document}

% *************** Włączenie definicji pierwszych stron ***************
\input{front.tex}


% *************** Część główna pracy ***************
\chapter*{Wstęp}

System obsługi parkingu to aplikacja umożliwiająca zarządzanie przestrzenią parkingową w sposób zautomatyzowany. System pozwala na rejestrowanie przyjazdów i odjazdów pojazdów, kontrolę zajętości miejsc oraz wizualizację stanu parkingu. Dzięki temu zapewnia przejrzystość w użytkowaniu parkingu oraz umożliwia szybkie wyszukiwanie informacji o pojazdach na podstawie numeru rejestracyjnego lub innych kryteriów.

Aplikacja obsługuje różne rodzaje pojazdów, takie jak motocykle, samochody osobowe i autobusy, które zajmują różną ilość miejsc na parkingu. Każdy pojazd posiada przypisane współrzędne zajmowanych pól oraz numer rejestracyjny. System rejestruje także historię zdarzeń, takich jak przyjazdy i odjazdy, co pozwala na późniejszą analizę ruchu na parkingu.

\
\addcontentsline{toc}{chapter}{Wstęp}
\newpage
% ********** Rozdział 1 **********
\chapter{Opis założeń projektu}
\section{Cele projetu}
%\subsection{Tytuł pierwszego podpunktu}

Celem projektu jest stworzenie systemu zarządzania parkingiem, który umożliwi efektywne monitorowanie i kontrolowanie zajętości miejsc parkingowych w zorganizowany sposób. System ma zapewniać:
Automatyzację procesów:
Obsługa przyjazdów i odjazdów pojazdów.
Automatyczna weryfikacja dostępności miejsc parkingowych w zależności od wymagań przestrzennych różnych typów pojazdów.
Wizualizację stanu parkingu:
Czytelna i intuicyjna reprezentacja graficzna zajętości parkingu w konsoli za pomocą znaków tekstowych.
Rejestrację zdarzeń:
Przechowywanie informacji o każdym zdarzeniu, takim jak przyjazdy i odjazdy pojazdów, wraz z datą i godziną.
Możliwość przeszukiwania historii na podstawie numeru rejestracyjnego pojazdu lub zakresu czasowego.
Wsparcie dla różnych typów pojazdów:
Obsługa motocykli, samochodów osobowych oraz autobusów, z uwzględnieniem różnej liczby miejsc zajmowanych przez te pojazdy.
Ułatwienie zarządzania przestrzenią parkingową:
Zwiększenie przejrzystości stanu zajętości miejsc.
Szybkie znajdowanie informacji o pojazdach zaparkowanych lub już odjeżdżających.
Dzięki wdrożeniu tego systemu możliwe jest zwiększenie wydajności operacyjnej parkingu, zminimalizowanie błędów w alokacji miejsc oraz ułatwienie pracy personelu zarządzającego parkingiem. System jest również skalowalny i może zostać zaadaptowany do różnej wielkości parkingów oraz dodatkowych wymagań użytkowników.

\end{itemize}


\section{Wymagania funkcjonale i niefunkcjonalne}

\noindent \textbf{Wymagania funkcjonalne}
\begin{itemize}
    \item Wymagania funkcjonalne
Zarządzanie pojazdami:

\item Możliwość dodawania pojazdu na parking z określeniem numeru rejestracyjnego oraz wymagań przestrzennych (typ pojazdu).
\item Usuwanie pojazdu z parkingu na podstawie numeru rejestracyjnego.
\item Walidacja dostępności miejsca:

\item Weryfikacja, czy wskazane miejsce parkingowe jest dostępne dla danego typu pojazdu.
\item Blokada dodawania pojazdu, jeśli miejsca są zajęte lub wykraczają poza granice parkingu.
\item Wizualizacja parkingu:

\item Wyświetlanie aktualnego stanu parkingu w konsoli w formie graficznej (np. znaki tekstowe reprezentujące zajętość miejsc: M dla motocykli, C dla samochodów, B dla autobusów).
Historia zdarzeń:

\item Rejestrowanie informacji o każdym przyjeździe i odjeździe pojazdu, w tym numeru rejestracyjnego, typu pojazdu, zajętych miejsc oraz daty i godziny zdarzenia.
Wyszukiwanie pojazdów i zdarzeń:

Możliwość wyszukiwania informacji o pojazdach w historii zdarzeń na podstawie:
Numeru rejestracyjnego.
Daty i godziny przyjazdu lub odjazdu.

\end{itemize}

\noindent \textbf{Wymagania niefunkcjonalne }
\begin{itemize}

\item System powinien działać płynnie przy obsłudze parkingów o różnej wielkości (od kilkunastu do kilkuset miejsc).

\item Skalowalność:
Projekt powinien umożliwiać łatwe dostosowanie do większych parkingów lub nowych typów pojazdów.
Czytelność interfejsu:

\item Wizualizacja parkingu powinna być czytelna i intuicyjna dla użytkownika, umożliwiając szybkie zrozumienie aktualnego stanu zajętości miejsc.
Niezawodność:

\item System powinien zapobiegać błędom, takim jak umieszczanie dwóch pojazdów na tym samym miejscu lub dodawanie pojazdu poza granice parkingu.
Przechowywanie danych:

\item Historia zdarzeń powinna być przechowywana w sposób trwały (np. w pamięci programu podczas działania lub w pliku).
Łatwość rozbudowy:

\item Kod powinien być modularny i dobrze udokumentowany, aby umożliwić łatwe dodawanie nowych funkcjonalności lub modyfikację istniejących.
Użyteczność:

\item System powinien być prosty w obsłudze, zarówno dla operatorów, jak i dla osób zarządzających parkingiem.
Bezpieczeństwo danych:

\item Numer rejestracyjny i inne informacje powinny być przechowywane w sposób bezpieczny, zgodnie z wymogami ochrony danych (np. w kontekście RODO, jeśli dotyczy).

\end{itemize}
 




% ********** Koniec rozdziału **********

\newpage
% ********** Rozdział 2 **********
\chapter{Opis struktury projektu}
\section{Podstawowe informacje techniczne}
Aplikacja została napisana w języku C\# i uruchamiana jest w konsoli.

\section{Minimalne wymagania sprzętowe}
\begin{itemize}
    \item Procesor minimum 2 GHz.
    \item 2 GB pamięci RAM.
    \item System operacyjny Windows/Linux.
    \item Zainstalowane środowisko .NET.
\end{itemize}

\section{Hierarchia klas}

System składa się z hierarchii klas, które reprezentują różne typy pojazdów oraz logikę zarządzania parkingiem. 
Podstawową klasą jest \textbf{Pojazd}, która jest klasą abstrakcyjną i stanowi bazę dla klas: \textbf{Motocykl}, \textbf{Samochod} oraz \textbf{Autobus}. 
Klasa \textbf{Parking} odpowiada za zarządzanie pojazdami i przechowywanie ich stanu. 
Program steruje aplikacją i zapewnia interakcję z użytkownikiem.

\section{Diagram klas}

Poniżej przedstawiono diagram klas UML obrazujący relacje między klasami w systemie:

\begin{itemize}
    \item \textbf{Program} – klasa główna, odpowiedzialna za uruchomienie aplikacji i zarządzanie logiką programu.
    \item \textbf{Parking} – klasa zarządzająca stanem parkingu, przechowująca informacje o zajętych miejscach oraz umożliwiająca dodawanie i usuwanie pojazdów.
    \item \textbf{Pojazd} – klasa bazowa dla różnych typów pojazdów, przechowuje informacje o numerze rejestracyjnym, typie pojazdu oraz zajmowanych miejscach parkingowych.
    \begin{itemize}
        \item \textbf{Motocykl} – klasa dziedzicząca po klasie Pojazd, reprezentująca motocykl na parkingu.
        \item \textbf{Samochód} – klasa dziedzicząca po klasie Pojazd, reprezentująca standardowy samochód osobowy.
        \item \textbf{Autobus} – klasa dziedzicząca po klasie Pojazd, zajmująca większą liczbę miejsc na parkingu.
    \end{itemize}
\end{itemize}

\subsection{Schemat klas}
Powyżej przedstawiony jest diagram klas obrazujący zależności między poszczególnymi elementami systemu.

\begin{figure}[]
    \centering
    \includegraphics[width=\textwidth]{diagram_klas2.jpg}
    \caption{Diagram klas systemu parkingowego}
    \label{fig:diagram_klas}
\end{figure}






% ********** Koniec rozdziału **********

\newpage
% ********** Rozdział 3 **********
\chapter{Harmonogram realizacji projektu}

\section{Harmonogram}
\begin{figure}[ht]
    \centering
    \includegraphics[width=1\linewidth]{harmonogram.jpg}
    \caption{Diagram Gantta}
\end{figure}
Realizacja projektu, przedstawiona na powyższym diagramie Gantta, obejmowała następujące etapy:
\begin{itemize}
    \item określenie założeń projektu - 26-27.10.2024,
    \item prace programistyczne - 07-17.02.2025,
    \item opracowanie dokumentacji projektu - 17-20.02.2025,
    \item prezentację i obronę projektu - 21.02.2025.
\end{itemize}

\section{Repozytorium i system kontroli wersji}
Kod projektu został przygotowany przy użyciu systemu kontroli wersji Git, a hostowany jest na platformie GitHub.
Repozytorium jest dostępne pod adresem: \url{https://github.com/3Czester/Programowanie_Casper_Czech_w71019}.


% ********** Koniec rozdziału **********
\newpage
% ********** Rozdział 4 **********
\chapter{Prezentacja warstwy użytkowej projektu}

Aplikacja do zarządzania parkingiem pozwala na wykonywanie operacji związanych z dodawaniem, usuwaniem oraz wyszukiwaniem pojazdów, a także wizualizację stanu parkingu. Poniżej przedstawiono zrzuty ekranu wraz z opisem poszczególnych funkcjonalności.

\section{Dodawanie pojazdu}
Dodanie pojazdu do systemu odbywa się poprzez podanie jego danych, takich jak numer rejestracyjny, typ pojazdu oraz miejsce parkingowe. Na poniższym zrzucie ekranu przedstawiono interfejs umożliwiający dodanie pojazdu.
\begin{figure}[H]
    \centering
    \includegraphics[width=0.8\linewidth]{Dodaj_pojazd.jpg}
    \caption{Dodawanie pojazdu do systemu}
    \label{fig:enter-label}
\end{figure}

\section{Wyświetlanie stanu parkingu}
Aplikacja umożliwia wizualizację aktualnego stanu parkingu, przedstawiając zajęte i wolne miejsca. Użytkownik może w łatwy sposób sprawdzić, które miejsca są dostępne.


\begin{figure}[H]
    \centering
    \includegraphics[width=0.8\linewidth]{Pokaż_parking.jpg}
    \caption{Widok parkingu w systemie}
    \label{fig:enter-label}
\end{figure}
\section{Usuwanie pojazdu}
Gdy pojazd opuszcza parking, użytkownik może usunąć go z systemu, co automatycznie zwalnia zajmowane miejsce.

\begin{figure}[H]
    \centering
    \includegraphics[width=0.8\linewidth]{Usuń_pojazd.jpg}
    \caption{Usuwanie pojazdu z systemu}
\end{figure}

\section{Wyszukiwanie pojazdu}
W celu szybkiego odnalezienia pojazdu system umożliwia jego wyszukiwanie po numerze rejestracyjnym. 

\begin{figure}[H]
    \centering
    \includegraphics[width=0.8\linewidth]{Znajdz_pojazd.jpg}
    \caption{Wyszukiwanie pojazdu}
\end{figure}

\section{Zamykanie aplikacji}
Użytkownik ma możliwość zamknięcia aplikacji poprzez odpowiednią opcję w menu.

\begin{figure}[H]
    \centering
    \includegraphics[width=0.8\linewidth]{Wyjdz.jpg}
    \caption{Zamykanie aplikacji}
\end{figure}





% ********** Koniec rozdziału **********

\newpage
% ********** Rozdział 5 **********
\chapter{Podsumowanie}

W ramach projektu wykonałem aplikację konsolową do zarządzania systemem parkingowym, wykorzystując język C\# oraz technologię .NET.

Realizacja projektu obejmowała następujące zadania:
\begin{itemize}
    \item określenie założeń funkcjonalnych i niefunkcjonalnych,
    \item implementację kluczowych funkcjonalności aplikacji, w tym:
    \begin{itemize}
        \item obsługę różnych typów pojazdów i ich miejsc parkingowych,
        \item wizualizację stanu parkingu w konsoli,
        \item rejestrowanie przyjazdów i odjazdów pojazdów,
    \end{itemize}
    \item stworzenie dokumentacji projektowej,
    \item prezentację i obronę projektu.
\end{itemize}

W ramach dalszego rozwoju planuję dodać następujące funkcjonalności usprawniające działanie aplikacji:
\begin{itemize}
    \item rozszerzenie obsługi o więcej typów pojazdów,
    \item integrację z bazą danych w celu trwałego przechowywania informacji o miejscach parkingowych,
    \item dodanie interfejsu graficznego w celu lepszego zobrazowania zajętych miejsc,
    \item wprowadzenie mechanizmu rezerwacji miejsc parkingowych.
\end{itemize}




% ********** Koniec rozdziału **********


\newpage
\input{R6.tex}
\newpage

% *************** Bibliografia ***************
\begin{thebibliography}{9}
\addcontentsline{toc}{chapter}{Bibliografia}

\bibitem{csharp_podstawy} Piotr Wróblewski, {\it C# 10.0. Wprowadzenie do programowania}, Wydawnictwo Helion, Gliwice 2022.  
\bibitem{csharp_zaawansowane} Krzysztof Foltak, {\it C#. Praktyczny kurs. Wydanie IV}, Wydawnictwo Helion, Gliwice 2021.  
\bibitem{csharp_wzorce} Leszek Grzesiak, {\it Wzorce projektowe w C#}, Wydawnictwo Helion, Gliwice 2020.  
\bibitem{programowanie_obiektowe} Grzegorz Dunikowski, {\it Programowanie obiektowe w języku C#}, Wydawnictwo Helion, Gliwice 2019.  
\bibitem{bazy_danych} Marek Konieczny, {\it Bazy danych i ich integracja z C#}, Wydawnictwo Helion, Gliwice 2021.  
\bibitem{git} Rafał Kuć, {\it Git. Rozproszony system kontroli wersji}, Wydawnictwo Helion, Gliwice 2020.  
\bibitem{zarzadzanie_projektami} Roman Kotapski, {\it Zarządzanie projektami. Metody, narzędzia, techniki}, Wydawnictwo PWE, Warszawa 2018.  
\bibitem{smart_parking} Tomasz Górski, {\it Inteligentne systemy transportowe}, Wydawnictwo Naukowe PWN, Warszawa 2022.  

\end{thebibliography}
\newpage


% *************** Zakończenie ***************
\chapter{Literatura}
\begin{itemize}
    \item Dokumentacja języka C\# - \url{https://learn.microsoft.com/en-us/dotnet/csharp/}
    \item Dokumentacja .NET - \url{https://dotnet.microsoft.com/}
\end{itemize}

\end{document}
% *************** Koniec pliku szablon.tex ***************
