% ********** Rozdział 1 **********
\chapter{Opis założeń projektu}
\section{Cele projetu}
%\subsection{Tytuł pierwszego podpunktu}

Celem projektu jest stworzenie systemu zarządzania parkingiem, który umożliwi efektywne monitorowanie i kontrolowanie zajętości miejsc parkingowych w zorganizowany sposób. System ma zapewniać:
Automatyzację procesów:
Obsługa przyjazdów i odjazdów pojazdów.
Automatyczna weryfikacja dostępności miejsc parkingowych w zależności od wymagań przestrzennych różnych typów pojazdów.
Wizualizację stanu parkingu:
Czytelna i intuicyjna reprezentacja graficzna zajętości parkingu w konsoli za pomocą znaków tekstowych.
Rejestrację zdarzeń:
Przechowywanie informacji o każdym zdarzeniu, takim jak przyjazdy i odjazdy pojazdów, wraz z datą i godziną.
Możliwość przeszukiwania historii na podstawie numeru rejestracyjnego pojazdu lub zakresu czasowego.
Wsparcie dla różnych typów pojazdów:
Obsługa motocykli, samochodów osobowych oraz autobusów, z uwzględnieniem różnej liczby miejsc zajmowanych przez te pojazdy.
Ułatwienie zarządzania przestrzenią parkingową:
Zwiększenie przejrzystości stanu zajętości miejsc.
Szybkie znajdowanie informacji o pojazdach zaparkowanych lub już odjeżdżających.
Dzięki wdrożeniu tego systemu możliwe jest zwiększenie wydajności operacyjnej parkingu, zminimalizowanie błędów w alokacji miejsc oraz ułatwienie pracy personelu zarządzającego parkingiem. System jest również skalowalny i może zostać zaadaptowany do różnej wielkości parkingów oraz dodatkowych wymagań użytkowników.

\end{itemize}


\section{Wymagania funkcjonale i niefunkcjonalne}

\noindent \textbf{Wymagania funkcjonalne}
\begin{itemize}
    \item Wymagania funkcjonalne
Zarządzanie pojazdami:

\item Możliwość dodawania pojazdu na parking z określeniem numeru rejestracyjnego oraz wymagań przestrzennych (typ pojazdu).
\item Usuwanie pojazdu z parkingu na podstawie numeru rejestracyjnego.
\item Walidacja dostępności miejsca:

\item Weryfikacja, czy wskazane miejsce parkingowe jest dostępne dla danego typu pojazdu.
\item Blokada dodawania pojazdu, jeśli miejsca są zajęte lub wykraczają poza granice parkingu.
\item Wizualizacja parkingu:

\item Wyświetlanie aktualnego stanu parkingu w konsoli w formie graficznej (np. znaki tekstowe reprezentujące zajętość miejsc: M dla motocykli, C dla samochodów, B dla autobusów).
Historia zdarzeń:

\item Rejestrowanie informacji o każdym przyjeździe i odjeździe pojazdu, w tym numeru rejestracyjnego, typu pojazdu, zajętych miejsc oraz daty i godziny zdarzenia.
Wyszukiwanie pojazdów i zdarzeń:

Możliwość wyszukiwania informacji o pojazdach w historii zdarzeń na podstawie:
Numeru rejestracyjnego.
Daty i godziny przyjazdu lub odjazdu.

\end{itemize}

\noindent \textbf{Wymagania niefunkcjonalne }
\begin{itemize}

\item System powinien działać płynnie przy obsłudze parkingów o różnej wielkości (od kilkunastu do kilkuset miejsc).

\item Skalowalność:
Projekt powinien umożliwiać łatwe dostosowanie do większych parkingów lub nowych typów pojazdów.
Czytelność interfejsu:

\item Wizualizacja parkingu powinna być czytelna i intuicyjna dla użytkownika, umożliwiając szybkie zrozumienie aktualnego stanu zajętości miejsc.
Niezawodność:

\item System powinien zapobiegać błędom, takim jak umieszczanie dwóch pojazdów na tym samym miejscu lub dodawanie pojazdu poza granice parkingu.
Przechowywanie danych:

\item Historia zdarzeń powinna być przechowywana w sposób trwały (np. w pamięci programu podczas działania lub w pliku).
Łatwość rozbudowy:

\item Kod powinien być modularny i dobrze udokumentowany, aby umożliwić łatwe dodawanie nowych funkcjonalności lub modyfikację istniejących.
Użyteczność:

\item System powinien być prosty w obsłudze, zarówno dla operatorów, jak i dla osób zarządzających parkingiem.
Bezpieczeństwo danych:

\item Numer rejestracyjny i inne informacje powinny być przechowywane w sposób bezpieczny, zgodnie z wymogami ochrony danych (np. w kontekście RODO, jeśli dotyczy).

\end{itemize}
 




% ********** Koniec rozdziału **********
